\documentclass[11pt]{article}

%%%%%%%%%%%%%%%%%%%%%%%%%%%%%%%%%%%%%%%%%%%%%%%%%%%%%%%%%%%%%%%%%%%%%%%%%%%%%%%%
% packages
%%%%%%%%%%%%%%%%%%%%%%%%%%%%%%%%%%%%%%%%%%%%%%%%%%%%%%%%%%%%%%%%%%%%%%%%%%%%%%%%

\usepackage{coling2020}
\usepackage{times}
\usepackage{url}
\usepackage{latexsym}
\usepackage{hyperref}
\hypersetup{
  colorlinks   = true, %Colours links instead of ugly boxes
  urlcolor     = blue, %Colour for external hyperlinks
  linkcolor    = blue, %Colour of internal links
  citecolor    = blue  %Colour of citations
}


%%%%%%%%%%%%%%%%%%%%%%%%%%%%%%%%%%%%%%%%%%%%%%%%%%%%%%%%%%%%%%%%%%%%%%%%%%%%%%%%
% paper configuration
%%%%%%%%%%%%%%%%%%%%%%%%%%%%%%%%%%%%%%%%%%%%%%%%%%%%%%%%%%%%%%%%%%%%%%%%%%%%%%%%

%\setlength\titlebox{5cm}
%\colingfinalcopy % Uncomment this line for the final submission

% You can expand the titlebox if you need extra space
% to show all the authors. Please do not make the titlebox
% smaller than 5cm (the original size); we will check this
% in the camera-ready version and ask you to change it back.


\title{Instructions for COLING-2020 Proceedings}

\author{First Author \\
  Affiliation / Address line 1 \\
  Affiliation / Address line 2 \\
  Affiliation / Address line 3 \\
  {\tt email@domain} \\\And
  Second Author \\
  Affiliation / Address line 1 \\
  Affiliation / Address line 2 \\
  Affiliation / Address line 3 \\
  {\tt email@domain} \\}

\date{}

%%%%%%%%%%%%%%%%%%%%%%%%%%%%%%%%%%%%%%%%%%%%%%%%%%%%%%%%%%%%%%%%%%%%%%%%%%%%%%%%
% latex functions
%%%%%%%%%%%%%%%%%%%%%%%%%%%%%%%%%%%%%%%%%%%%%%%%%%%%%%%%%%%%%%%%%%%%%%%%%%%%%%%%


%%%%%%%%%%%%%%%%%%%%%%%%%%%%%%%%%%%%%%%%%%%%%%%%%%%%%%%%%%%%%%%%%%%%%%%%%%%%%%%%
% document text
%%%%%%%%%%%%%%%%%%%%%%%%%%%%%%%%%%%%%%%%%%%%%%%%%%%%%%%%%%%%%%%%%%%%%%%%%%%%%%%%

\begin{document}
\maketitle
\begin{abstract}
\end{abstract}

%
% The following footnote without marker is needed for the camera-ready
% version of the paper.
% Comment out the instructions (first text) and uncomment the 8 lines
% under "final paper" for your variant of English.
% 
\blfootnote{
    %
    % for review submission
    %
    \hspace{-0.65cm}  % space normally used by the marker
    Place licence statement here for the camera-ready version. 
    %
    % % final paper: en-uk version 
    %
    % \hspace{-0.65cm}  % space normally used by the marker
    % This work is licensed under a Creative Commons 
    % Attribution 4.0 International Licence.
    % Licence details:
    % \url{http://creativecommons.org/licenses/by/4.0/}.
    % 
    % % final paper: en-us version 
    %
    % \hspace{-0.65cm}  % space normally used by the marker
    % This work is licensed under a Creative Commons 
    % Attribution 4.0 International License.
    % License details:
    % \url{http://creativecommons.org/licenses/by/4.0/}.
}

\section{Introduction}
\label{sec:intro}

\section{Method Overview}
\label{sec:method}

\section{Experiments}
\label{sec:experiments}

\section{Related Work}
\label{sec:related}

\subsection{Ben}

Very similar to what we want to do: \cite{rupnik2016news,miranda2018multilingual,wang2018estimation,germann2019scalable,seki2018exploring,seki2020cross,linger2020batch}

Brexit case study: \cite{peterlin2019detecting}

Also similar, but involving search terms; probably just good for citations and not techniques: \cite{rupnik2016news}

Multilingual BERT:

\cite{K2020Cross-Lingual}

\cite{pires2019multilingual}

More multilingual models non-BERT from google:

https://ai.googleblog.com/2019/07/multilingual-universal-sentence-encoder.html

https://arxiv.org/abs/1807.11906
https://arxiv.org/abs/1810.12836
https://arxiv.org/abs/1902.08564
https://arxiv.org/abs/1906.08401
https://arxiv.org/abs/1907.04307

Old survey but good on non deep learning techniques of the era: \cite{oard1998survey}.

\subsection{Stefanos}

Datasets:

\cite{liu2019dens}

\cite{SemEval2018Task1}

Pretrained models:

\cite{puri2018large}

\cite{kant2018practical}

Examples of large scale sentiment analysis:

\cite{mohammad2015sentiment}

\cite{hemmatian2017survey}

\cite{yang2015twitter}

\subsection{Nate}

Look at the way these papers run experiments:

\cite{tifrea2018poincar,meng2019spherical}

Datasets at: \url{https://aclweb.org/aclwiki/Similarity_(State_of_the_art)}

Use wikidata to automatically generate classes: https://www.wikidata.org/wiki/Q43689 
https://pywikidata.readthedocs.io/en/latest/

wikidata church fathers in latin: \url{https://query.wikidata.org/#SELECT%20%3Fperson%20%3FpersonLabel%0AWHERE%20%7B%0A%20%20%3Fperson%20wdt%3AP361%20wd%3AQ182603.%0A%20%20SERVICE%20wikibase%3Alabel%20%7B%20bd%3AserviceParam%20wikibase%3Alanguage%20%22la%22.%20%7D%0A%7D}

SPARQL Tutorial: https://www.wikidata.org/wiki/Wikidata:SPARQL_tutorial

Can we use wikidata to automatically generate good evaluations?

\section{Discussion}
\label{sec:discussion}


\bibliographystyle{coling}
\bibliography{main}

\end{document}
